% Chem 719/749 Fall 2010 Problem Sets: PS 2

\documentclass[11pt]{article}

%\usepackage{overcite}
\usepackage{graphicx}
\usepackage{times}

\setlength{\textwidth}{6.5in}
\setlength{\topmargin}{0.0in}
\setlength{\headheight}{0.0in}
\setlength{\headsep}{0.0in}
\setlength{\textheight}{9.0in}
\setlength{\oddsidemargin}{0.0in}

\begin{document}

\thispagestyle{empty}
\renewcommand{\thefootnote}{\fnsymbol{footnote}}
\let\oldhat\hat
\renewcommand{\vec}[1]{\mathbf{#1}}
\renewcommand{\hat}[1]{\oldhat{\mathbf{#1}}}

\begin{minipage}[t]{5.0in}
\begin{center}
%
{\bfseries
University of South Carolina \\
Department of Chemistry \\
\vspace{1.0em} CHEM 749 Spring 2020: \\ 
Chemistry and Physics of Low-Dimensional Materials \\
Andrew Greytak \\
\vspace{1.5em}\Large{Supplement to Lecture 6: \\ Graphene and Carbon Nanotube Band Structure from Tight Binding}}
\end{center}
\end{minipage}
\hfill
\begin{minipage}[t]{1.0in}
\raisebox{0.0em}{\raisebox{-\height}{
\includegraphics[width=\textwidth]{USC_Stan_Pos_202_logo_only_clear_bkgd.png}
}}\\
\vfill
\end{minipage}

\setcounter{page}{1}
% \renewcommand{\baselinestretch}{1.5} \small\normalsize
\renewcommand{\labelenumi}{\bfseries\arabic{section}\alph{enumi}.}


\vspace{1.0em}
\noindent\hrulefill

\section*{Tight-binding result}

The tight-binding calculation of the graphene band structure (see
Wallace 1947 \cite{wallace_band_1947}, Saito et al. 1992 \cite{saito_electronic_1992}, Castro Neto et al. 2009 \cite{castro_neto_electronic_2009}) is more complicated than that for our
1-D $s$-orbital chain both because it's in two dimensions, and because
there are two atoms per unit cell (two atoms per Bravais lattice site). This two-atom {\em basis} means there are effectively two hexagonal lattices formed by two sets of atoms $A$ and $B$. Bloch wavefunctions $\psi_A$ and $\psi_B$ describe states formed from the $p_z$ orbitals on atoms in lattices $A$ and $B$ ($\phi_A$ and $\phi_B$), but the actual band states are linear combinations of these:
\begin{eqnarray}
\psi_{A,k} = \sum_A e^{i \vec{k} \cdot \vec{R}_A} \phi_A(\vec{r}-\vec{R}_A) \\
\psi_{B,k} = \sum_B e^{i \vec{k} \cdot \vec{R}_B} \phi_B(\vec{r}-\vec{R}_B) \\
\Psi_k = c_{A,k} \psi_{A,k} + c_{B,k} \psi_{B,k} 
\end{eqnarray}
and we want to find coefficients $c_{i,k}$ such that
\begin{equation}
\vec{H} \Psi_k = E_k \Psi_k
\end{equation}
 To find the band energies $E_k$ according to the variational principle, we must solve the following ``secular determinant'':
\begin{equation}
\left| \begin{array}{ll} H_{AA} - E \, S_{AA} & H_{AB} - E\,  S_{AB} \\
	 H_{BA} - E\,  S_{BA} & H_{BB} - E \, S_{BB} \end{array} \right| = 0
\end{equation}
Where $H_{ij}$ and $S_{ij}$ represent the interaction and overlap integrals between each basis Bloch function. We require that $H_{AB}=H_{BA}^*$ and $S_{AB}=S_{BA}^*$. The problem is greatly simplified for graphene by noting that all of the atomic orbitals are the same so $H_{AA} = H_{BB} = H_{11}$. We can further simplify by neglecting overlap ($S_{AA}=S_{BB} = 1$ and $S_{AB}=0$). As described by Wallace, the nearest-neighbor correction to the atomic orbital energies then becomes:
\begin{eqnarray}
E_{\mathrm{2D}} & = & H_{11} \pm H_{12}  \label{e2d} \\
H_{11} & = & E_\circ -  \gamma_\circ^\prime \sum_{A,A = \mathrm{n.n.}} e^{i\vec{k}(\vec{R}_{A^\prime}-\vec{R}_{A})} \\
H_{BA} & = & \phantom{E_\circ + \ }-\gamma_\circ \sum_{A,B = \mathrm{n.n.}} e^{i\vec{k}(\vec{R}_B-\vec{R}_A)} \\
H_{12} & = & [H_{BA}H_{BA}^*]^{1/2}
\end{eqnarray}
\begin{figure}[h]
\center\includegraphics[width=4.5in]{graphene_bands_fig1.png}
\caption{{\small {\em Left:} The graphene lattice. The two groups of
like atoms are differently shaded. Representative primitive lattice
vectors and nearest-neighbor vectors are shown. {\em Right:}
Reciprocal space. The black dots describe reciprocal lattice
points. The black line marks the extent of the first Brillouin
zone. The colored circles near the $\vec{K}$ and $\vec{K}^\prime$
points help to identify regions that describe the same states: Each
filled red circle describes the same state. The filled and open
circles of each color describe two sets of states that are linked by
inversion symmetry. It can be seen that the Brillouin zone includes
exactly one instance of each of these points, completely describing
the region surrounding each of the two distinct $\vec{K}$ points.}}
\label{fig1}
\end{figure}
where $H_{11}$ accounts for nearest neighbors within the same Bravais lattice (i.e. two bonds away, and includes an on-site term) and $H_{12}$ accounts for nearest neighbors in different Bravais lattices (i.e. one bond away). In the crudest treatment, $H_{11}$ can be neglected. The $+$ and $-$ in Equation \ref{e2d} refer to the valence and
conduction bands respectively, and $-\gamma_\circ$ and $-\gamma_\circ^\prime$ are the
nearest-neighbor interaction integrals (written such that $\gamma_\circ$ and $\gamma_\circ^\prime$ have positive values, which follows the convention used by Wallace and Geim). Inspection of the graphene lattice
 shows that for a given atom, there are six nearest-neighbor Bravais lattice vectors $\vec{R}_{AA^\prime}$. In what follows, $a$ is the {\bf bond length}, so that $a \sqrt{3}$ is length of a primary lattice vector (the lattice constant). This is the convention of Geim {\em et al.} The $\vec{R}_{AA^\prime}$ are:
\begin{eqnarray}
\vec{R}_{A^\prime}-\vec{R}_A & = & \pm \sqrt{3} a\vec{\hat{y}} \\
\vec{R}_{A^\prime}-\vec{R}_A & = & \pm \frac{3}{2}a\vec{\hat{x}} \mp \frac{\sqrt{3}}{2}a\vec{\hat{y}} \\
\vec{R}_{A^\prime}-\vec{R}_A & = & \pm \frac{3}{2}a\vec{\hat{x}} \mp \frac{\sqrt{3}}{2}a\vec{\hat{y}} \\
\end{eqnarray}
while there are three nearest-neighbor vectors $\vec{R}_{AB}$ between atoms on the two different sites:
\begin{eqnarray}
\vec{R}_B-\vec{R}_A & = & a\vec{\hat{x}} \\
\vec{R}_B-\vec{R}_A & = & -\frac{a}{2}\vec{\hat{x}} + \frac{\sqrt{3}}{2}a\vec{\hat{y}} \\
\vec{R}_B-\vec{R}_A & = & -\frac{a}{2}\vec{\hat{x}} - \frac{\sqrt{3}}{2}a\vec{\hat{y}} \\
\end{eqnarray}
Then,
\begin{eqnarray}
H_{11} & = & E_\circ - \gamma_\circ^\prime \left[ 2\cos (k_y \sqrt{3} a) + 4 \cos (k_x \frac{3}{2} a) \cos (k_y \frac{\sqrt{3}}{2} a) \right] \\
H_{BA} & = & \gamma_\circ \left[ e^{i{k_x a}} + 2 e^{-ik_x a/2} \cos( k_y \frac{\sqrt{3}}{2}a)     \right] \\
H_{12}  =   \left[ H_{BA}^* H_{BA} \right]^{1/2}  & = & \gamma_\circ \left[ 1 + 4 \cos (k_x \frac{3}{2}a ) \cos (k_y \frac{\sqrt{3}}{2} a) + 4 \cos^2 (k_y \frac{\sqrt{3}}{2}a) \right]^{1/2}
\label{bands}
\end{eqnarray}
It's possible to write these more concisely as
\begin{eqnarray}
H_{11} = E_\circ - \gamma_\circ^\prime f(\vec{k}) \\
H_{12} = - \gamma_\circ \sqrt{3 + f(\vec{k})}
\end{eqnarray}
with
\begin{equation}
f(\vec{k}) = 2\cos (k_y \sqrt{3} a) + 4 \cos (k_x \frac{3}{2} a) \cos (k_y \frac{\sqrt{3}}{2} a)
\end{equation}
Importantly, this works out to $E_\mathrm{2D} = E_\circ \pm 3 \gamma_\circ - 6 \gamma_\circ^\prime$ at
$\vec{k} = 0$, where the $+$ version represents the upper (conduction) band and the $-$ version the lower (valence band). The bands are degenerate for
$\vec{k}$'s at the six corners of the Brilloiun zone with $E_\mathrm{2D} = E_\circ + 3 \gamma_\circ^\prime$  (for example, try
$\vec{k} = \frac{4}{3\sqrt{3}}\frac{\pi}{a}\vec{\hat{y}}$). The Brillouin zone
contacts six such points, which are typically labeled K. We expect $\gamma_\circ > \gamma_\circ^\prime$ because the AB distances are shorter, and in fact the essential behavior can be understood by considering $\gamma_\circ^\prime = 0$ (ignoring $H_{11}$).

We can plot this band structure as a three-dimensional surface, with
energy on the vertical scale. The fact that the points of contact
between the valence and conduction band fall at the edge of the
Brillouin zone make it tricky to visualize, and count, the regions of
$\vec{k}$ space that are close to the band edge in this system. It
will be easiest to see the form of the conical band structure around
the K points if we use the ``extended zone'' scheme (continuing the
band structure outside of the zone). This is perfectly fine as long as
we remember that $\vec{k}$ values separated by a reciprocal lattice
vector actually describe the same states. Thus there are evidently a
total of {\em two} distinct points of contact between the bands; these
points are labeled these $\vec{K}$ and $\vec{K}^\prime$ in Figure
\ref{fig1}. These are equivalent to each other by inversion symmetry,
but the regions around each describe different states.

\begin{figure}[t]
\center\includegraphics[width=5.5in]{graphene_bands_fig2.png}
\caption{{\small {\em Left:} Top view of the conduction band of
graphene, plotted with a color scale of energy. The Brillouin zone
edge is highlighted.  {\em Right:} A 3-D view of energy vs. $k_x$ and
$k_y$ illustrating the two bands and the conical nature of the contact
between bands at points $\vec{K}$ and $\vec{K}^\prime$.}}
\label{graphene}
\end{figure}


\section*{Quantum confinement in single-walled carbon nanotubes}

\begin{figure}[tbh]
\center\includegraphics[width=\textwidth]{graphene_bands_fig3.png}
\caption{{\small Allowed $\vec{k}$ vectors for a (5,2) nanotube. Each
line corresponds to states with constant $q$ and various values of
$t$; $t$ thus describes the wavevector along the nanotube axis. The
black dots in the view at left indicate $\vec{k}$ for $t=0$ for
several representative subbands. As seen at right, each line of
allowed $\vec{k}$ values describes a valence and conduction
sub-bands. The thicker segments of the lines indicate portions of each
subband that fall within the Brillouin zone and thus describe unique
states. Because the $\vec{K}$ and $\vec{K}^\prime$ points are included
among the allowed $\vec{k}$ values, this nanotube is expected to be a
metal.}}
\label{five_two}
\end{figure}

Now, we can go about imposing quantum confinement by declaring a
roll-up vector $\vec{C}_h = n \vec{a}_1 + m \vec{a}_2$ with
$\vec{a}_1$ and $\vec{a}_2$ primitive (direct) lattice vectors. The
length of $\vec{C}_h$ is the circumference of the tube; a trip around
the circumference must comprise an integer number of electron
wavelengths, or in other words
\begin{equation}
\vec{C}_h \cdot \vec{k} = 2\pi q
\end{equation}
where $q$ is an integer. In other words, the component of $\vec{k}$ in
the $\vec{C}_h$ direction must be an integer multiple of $2
\pi/|\vec{C}_h|$. The component of $\vec{k}$ perpendicular to
$\vec{C}_h$ could have any value: we will call it $t$.
\begin{eqnarray}
\vec{k} & = & \frac{2 \pi q}{|\vec{C}_h|} \vec{\hat{C}}_h + t (\perp\vec{\hat{C}}_h) \\
\vec{\hat{C}}_h & = & \frac{C_{h,x}}{|\vec{C}_h|} \vec{\hat{x}} + \frac{C_{h,y}}{|\vec{C}_h|} \vec{\hat{y}} \\
\perp\vec{\hat{C}}_h & = & - \frac{C_{h,y}}{|\vec{C}_h|} \vec{\hat{x}} + \frac{C_{h,x}}{|\vec{C}_h|} \vec{\hat{y}} \\
\end{eqnarray}
where we have arbitrarily set $\perp\vec{\hat{C}}_h$ to be rotated to
the 90$^\circ$ to the left vs. $\vec{\hat{C}}_h$. Now, we have a
complete description of $\vec{k}$:
\begin{equation}
\vec{k}  =  \left(\frac{2 \pi q C_{h,x}}{|\vec{C}_h|^2} - t \frac{C_{h,y}}{|\vec{C}_h|} \right) \vec{\hat{x}} 
	+ \left(\frac{2 \pi q C_{h,y}}{|\vec{C}_h|^2} + t \frac{C_{h,x}}{|\vec{C}_h|} \right) \vec{\hat{y}}
\label{k_q_t}
\end{equation}

Using Equation \ref{k_q_t} we step through integer values of $q$, and
for each one, we generate a line of $\vec{k}$'s by varying $t$
continuously. At each $\vec{k}$, Equation \ref{bands} can be used to
compute the valence and conduction band energy. This is exactly what
is done to generate Figure \ref{five_two}, using $\vec{a}_1 = (3a/2)
\vec{\hat{x}} + (a\sqrt{3}/2) \vec{\hat{y}}$ and $\vec{a}_2 = (3a/2)
\vec{\hat{x}} - ( a\sqrt{3}/2) \vec{\hat{y}}$ as the primitive lattice
vectors.


The values of $t$ and $q$ can be positive, negative, or zero. For the
figure, we set the limits of $q$ and $t$ to completely cover the
limits of our plot. Of course, only values of $\vec{k}$ that actually
fall within the Brillouin zone will describe unique states. If
desired, this condition can be met by checking that:

\begin{eqnarray}
|k_x| & \leq & \frac{2}{3}\frac{\pi}{a} \\
\frac{1}{\sqrt{3}}|k_x| + |k_y| & \leq & \frac{4}{3\sqrt{3}}\frac{\pi}{a}
\end{eqnarray}

The ($n$,$m$) = (5,2) nanotube described in Figure \ref{five_two} is
metallic. Figure \ref{five_three} describes a (5,3) nanotube, which is
a semiconductor since the valence and conduction subbands never touch.

\begin{figure}[ht]
\center\includegraphics[width=\textwidth]{graphene_bands_fig4.png}
\caption{{\small Allowed $\vec{k}$ vectors for a (5,3)
nanotube. Because the $\vec{K}$ and $\vec{K}^\prime$ points are not
included among the allowed $\vec{k}$'s, this nanotube is expected to
be a semiconductor. Note that the band-edge states are still close to
the $\vec{K}$ and $\vec{K}^\prime$ points, and that the nanotube is a
direct bandgap semiconductor since the conduction subband minimum and
valence subband maximum occur at the same value of $\vec{k}$.}}
\label{five_three}
\end{figure}

There are several ways to plot the dispersion of the subbands along
the nanotube axis. In Figure \ref{subbands}, the energy of the various
subbands versus axial wavevector $t$ is shown for the (5,2) and (5,3)
examples. Looking carefully at the metallic (5,2) case, there are {\em
two} unique subbands that are partially filled at the Fermi energy
(remember $E_\mathrm{F}=0$ here) ... for some value of $E$ close to
zero, each of these subbands intersects $E$ twice (once for $t>0$ and
once for $t<0$, corresponding to states propagating forwards and
backwards along the tube axis) leading to a total of four states with
$E=E_\mathrm{F}$. A similar situation is found in the semiconducting
tube: there are two distinct conduction band minima and two distinct
valence band maxima within the Brillouin zone, so if $E_\mathrm{F}$ is
brought close to $E_\mathrm{C}$ or $E_\mathrm{V}$, there will always
be two subbands that are equally populated with carriers. Thus, in the
strictest sense, carbon nanotubes are not one-dimensional electronic
systems since there are always at least two 1-D subbands that
participate! Nonetheless, the reduced dimensionality versus bulk
graphite is keenly evidenced in their electronic properties.

\begin{figure}
\center\includegraphics[width=5.5in]{graphene_bands_fig5.png}
\caption{{\small Dispersion $E$ versus $t$ for the (5,2) and (5,3)
nanotubes discussed above. The different subbands are all plotted on
the same axes, with portions of the subbands that fall within the
Brillouin zone marked by thicker lines.}}
\label{subbands}
\end{figure}



\bibliography{chem719}
\bibliographystyle{cv.bst}



\end{document}
